\chapter{Introduction}
\label{chap:intro}
\chaptermark{Optional running chapter heading}

\section{Background}
A large amount of data generated today is unlabeled, which complicates the extraction of useful information. Contrastive learning, a type of unsupervised learning, has been effective in leveraging such data, particularly in areas where labeled data is scarce or expensive to obtain. Existing frameworks like SimCLR \cite{simclr} and Momentum Contrast (MoCo) \cite{moco} have shown potential but often struggle with large batch sizes and hyperparameter sensitivity.

Current methods face limitations with scalability and robustness, especially when applied to complex datasets with varying characteristics. These issues highlight the need for more adaptable and efficient learning models.

\section{Research Questions}
This thesis conducts a comparative analysis of modern techniques for 2D data visualization within the framework of unsupervised learning. It examines whether integrating the Area Under the Curve (AUC) metric \cite{sharma2023auc} with contrastive learning frameworks can address their known limitations and improve their performance across diverse datasets.

\section{Experimental Approach}
The research employs a comparative experimental design using different datasets to evaluate the performance of enhanced contrastive learning methods against traditional approaches. The focus is on accuracy and robustness in visualizing data.

\section{Results and Conclusions}
Results show that adding the AUC metric to contrastive learning frameworks increases their robustness, particularly in limited-resource settings. Also, the adapted framework, t-SimCNE with AUC-CL, provides more clear and interpretable visual representations. For example, The initial approach used a batch size of at least 1024, while we make it possible to achieve similarly good results with a batch size of 256.
These improvements are significant for data analysis in areas with limited labeled data, potentially improving decision-making and advancing research in various fields.

\section{Thesis Structure}
The structure of our paper is as follows:

\begin{itemize}
    \item Chapter 2: Literature Review – Reviews theoretical background and previous research in unsupervised and contrastive learning.
    \item Chapter 3: Methodology – Describes the experimental methodologies, data sources, and specific frameworks used.
    \item Chapter 4: Results and Analysis – Discusses the experimental results and analyzes the findings.
    \item Chapter 5: Discussion – Interprets the results, discusses implications and limitations, and suggests future research directions.
    \item Chapter 6: Conclusion – Summarizes the study's findings and contributions and outlines future research possibilities.
\end{itemize}