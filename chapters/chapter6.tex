\chapter{Conclusion}
\label{chap:conclusion}
\section{Summary of Findings}

In  this thesis we implemented various advanced unsupervised learning techniques for 2D data visualization and conducted a comparative analysis. Our study focused primarily on frameworks such as SimCLR, t-SimCNE, and AUC-CL, testing these methodologies across diverse datasets including CIFAR-10, Leukemia, Bloodmnist, and Dermamnist.

The findings of this research underscore several key points:
\begin{enumerate}
    \item {Enhanced Visualization Capabilities:} Among the tested methodologies, t-SimCNE combined with the AUC-CL framework provided superior visualization results while using smaller batch sizes. 
    \item {Robustness to Batch Size Variations:} An important feature of the AUC-CL framework is its robustness against batch size variations, making it suitable for scenarios with computational resource constraints.
    \item {Quantitative Metrics:}  The use of metrics, like k-NN classifier accuracy, confirmed that the embeddings from t-SimCNE augmented with AUC-CL have better quality.
    
\end{enumerate}

\section{Future Work}

Further research can build on the work of this thesis by looking for better ways to handle large amounts of data and improve our algorithms. Additionally, developing tools and hard negative sampling techniques for selecting informative negative examples could enhance these methods. More so, future studies can try using our current solutions on more complex datasets and with more computational power for longer periods, which could help produce much clearer and more clustered visual representations of data. 

\section{Concluding Remarks}

The research conducted for this thesis advances the understanding and application of contrastive learning techniques for 2D data visualization. By developing and implementing robust, resource-efficient strategies, this work contributes to making data visualization techniques more accessible and practical across various domains. The further refinement and adaptation of these methods are important in addressing the evolving challenges within the field of artificial intelligence.
\ldots