\begin{abstract}

Modern datasets nowadays contain a significant amount of unlabeled data, presenting challenges in deriving valuable insights. Recent advancements in contrastive learning techniques, such as SimCLR and t-SimCNE, have demonstrated efficacy in this context, particularly for visualizing large datasets and extracting meaningful insights. Despite their effectiveness, these methods often struggle with the requirements for large batch sizes and extensive computational resources, and they may fail to provide clear and distinct 2D visualizations for certain types of data.

To address these limitations, this thesis introduces optimization techniques, specifically AUC-CL and Hard Negative Sampling, to enhance the existing t-SimCNE framework. A comparative analysis of the modified algorithm is conducted, assessing its effectiveness through 2D graphical representations and evaluating its performance across various batch sizes using the k-nearest neighbors (k-NN) metric.
\end{abstract}